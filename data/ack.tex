% 如果使用声明扫描页,将可选参数指定为扫描后的 PDF 文件名,例如:
% \begin{acknowledgement}[scan-statement.pdf]
\begin{acknowledgement}
  本科四年,午梦千山,窗阴一箭,其间在学习科研方面得到了很多老师的教诲和同学的帮助。首先要感谢课题组的周青老师,是周老师当年生动详实的讲座引发了我对汽车安全领域的浓厚兴趣,难以忘怀周老师给我们讲述自己如何是用一个十分简洁的力学模型做出了关于乘员头部碰撞防护的工作,也难以忘怀每一次和周老师交流时的侃侃而谈。 桃李不言,下自成蹊。 同时,也十分感谢课题组的夏勇老师,十分感谢夏老师的悉心指导,在我做毕业设计期间指导我对于这一领域有了初步的理解并使得我掌握了很多分析处理相关问题的能力。\\
  \indent感谢冯西桥老师的固体力学基础课程和许春晓老师的流体力学课程,在他们的课堂上我建立了关于经典力学体系的初步认识和系统概念。也十分感谢张雄老师的有限元基础和任玉新老师的计算流体力学基础,正是这两门数值模拟的课程让我掌握了力学建模的能力,能进一步对诸多问题进行尝试和研究。\\
  \indent在力学专业课程之外,十分感谢钱学森力学班对于课程灵活性的包容和鼓励,使我能学习一些自己感兴趣的课程,而这些课程也教会了我相当多的东西。最最令我难忘的是高研院徐湛老师开设的量子力学的课程,课堂上的精妙讲述,课堂下对于诸多问题的激烈讨论使得我深深地爱上了物理学,不仅仅是量子力学的美妙和神奇,徐老师的严谨治学和深刻思维给我至今依然是我心中的榜样。\\
  \indent 从大二开始在曹炳阳老师实验室做研究的两年时间是紧张而快乐的,每周三的组会对一个个问题的讲解和讨论使得我逐渐确立了一套自己的关于研究的体系性的看法。 其间,感谢实验室的华钰超师兄,带着我学会了声子蒙特卡洛模拟、第一性原理模拟的模拟方法,并在讨论中给我详细而有前瞻的指导。也感谢东京大学的鞠生宏师兄给予我的关于界面格林函数乃至相关优化算法的研究的指导。\\
  \indent 感谢东京大学Junichiro Shiomi教授,在实验室暑研期间,每次和教授的讨论,他总能把握问题的核心和关键,深厚的理论基础和丰富的科研经历,特别是那随手捻来的文献和独到深刻的见解都让我叹为观止。依稀记得,在最终的汇报中老师给我讲述了面对详实的数据给我讲述了不应该拘泥于分析工具而应该着眼于研究目标和物理本质的研究思路,我还依稀记得在和他讨论时他所说:
  \begin{quote}
   'You are the first one in the world to gain these data, but what you should care about is not only the application, for example, use ML to gain the Minimum ITC, I strongly recommend you that you should consider the physics using something like regression. You know, regression is a very old tool but you need remember regression can but the Machine Learning cannot tell you, is the physics. '
  \end{quote}
  \indent
  \indent感谢实验室潘哲鑫师兄给予我在实验方法和分析上的指导,感谢罗海灵师兄、陈冠华师兄给我在电池分析和建模上的指导。\\
  \indent感谢刘千惠同学和我讨论实验设计和建模分析的方法、以及图像分析的算法,最重要的是一直以来对我遇到困难和挫折的无尽支持和鼓励,当然还有每一天给我带来的快乐和欣喜。 我曾以为人生的意义在于四处游荡逃亡,但我如今已找到愿意驻足的地方。
  \\
  \indent 感谢我的父母一直以来的付出和对我的支持理解,他们总是给我最大限度的自由,坚定地支持我选择和做出的每一个决定。
  \\
  \indent 感谢和我一起度过这四年的酸甜时光的同学和朋友,是你们组成了我大学时光中最明丽的亮色。
  \\
  \indent感谢计算机系薛瑞尼同学提供的论文的Latex模板,给我论文的撰写提供了很大的方便。
  \\
  \indent 愿这篇论文成为审慎思考的起点,愿我能登高而望,星空璀璨。
  \\
  \indent 夢のある人が、周りに流されない。
  \\
  \indent 有梦想,你就不会随波逐流。
  \end{acknowledgement}
