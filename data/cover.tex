\thusetup{
  %******************************
  % 注意:
  %   1. 配置里面不要出现空行
  %   2. 不需要的配置信息可以删除
  %******************************
  %
  %=====
  % 秘级
  %=====
  %secretlevel={秘密},
  %secretyear={10},
  %
  %=========
  % 中文信息
  %=========
  ctitle={锂离子电池主要界面力学特性测试和分子模拟研究},
  cdegree={工学学士},
  cdepartment={航天航空学院},
  cmajor={工程力学(钱学森力学班)},
  cauthor={肖飞宇},
  csupervisor={夏勇\quad副研究员},
  %cassosupervisor={陈文光教授}, % 副指导老师
  %ccosupervisor={某某某教授}, % 联合指导老师
  % 日期自动使用当前时间,若需指定按如下方式修改:
  % cdate={超新星纪元},
  %
  % 博士后专有部分
  %cfirstdiscipline={计算机科学与技术},
  %cseconddiscipline={系统结构},
  %postdoctordate={2009年7月——2011年7月},
  %id={编号}, % 可以留空: id={},
  %udc={UDC}, % 可以留空
  %catalognumber={分类号}, % 可以留空
  %
  %=========
  % 英文信息
  %=========
  etitle={The Testing/Characterization and Molecular Simulation of interfaces in Lithium-ion batteries},
  % 这块比较复杂,需要分情况讨论:
  % 1. 学术型硕士
  %    edegree:必须为Master of Arts或Master of Science(注意大小写)
  %             “哲学、文学、历史学、法学、教育学、艺术学门类,公共管理学科
  %              填写Master of Arts,其它填写Master of Science”
  %    emajor:“获得一级学科授权的学科填写一级学科名称,其它填写二级学科名称”
  % 2. 专业型硕士
  %    edegree:“填写专业学位英文名称全称”
  %    emajor:“工程硕士填写工程领域,其它专业学位不填写此项”
  % 3. 学术型博士
  %    edegree:Doctor of Philosophy(注意大小写)
  %    emajor:“获得一级学科授权的学科填写一级学科名称,其它填写二级学科名称”
  % 4. 专业型博士
  %    edegree:“填写专业学位英文名称全称”
  %    emajor:不填写此项
  edegree={Bachelor of Engineering},
  emajor={Mechanics},
  eauthor={Feiyu Xiao},
  esupervisor={Associate Professor Yong Xia},
  %eassosupervisor={Chen Wenguang},
  % 日期自动生成,若需指定按如下方式修改:
  % edate={December, 2005}
  %
  % 关键词用“英文逗号”分割
  ckeywords={电池材料断裂,活性层集流体脱层失效,混合拉伸/剪切加载,弹性常数},
  ekeywords={Fracture of the electrode material, debonding of electrode and substrate, combined tension/shear loading,elastic constants}
}

% 定义中英文摘要和关键字
\begin{cabstract}
研究表明,锂离子电池电极材料的断裂可以造成内部界面的接触失效,引发诸如SEI膜的形成和溶解等连锁反应,从而会进一步造成锂离子电池的性能衰退。同时,电极材料的断裂可能发生在不同的层次和尺度,如晶格、颗粒乃至整个活性层层次。另外,锂离子的脱嵌和嵌入在活性颗粒中产生的应力分布有可能引发颗粒的断裂并进一步在引起宏观上的断裂失效,如颗粒和胶层之间的断裂脱层。这些失效模式十分的复杂,要想对其进行研究和表征,需要考虑不同材料的特性和之间的相互作用乃至考量不同尺度和不同物理场之间的耦合作用。理解电极材料组分之间的界面相互作用,对于防止电极材料的断裂失效有着十分重要的意义。\\
\indent 一方面,在电动汽车的碰撞事故中,车载锂离子电池不可避免地会受到外部力学加载,常常会导致活性层与集流体之间的脱层失效从而引发电池的进一步失效。 实际上,活性层和集流体之前的粘接强度的强弱决定了随着电化学过程进行所产生的内部应力是否会引起内部破坏和失效,另外,粘接强度也显著地影响着接触内阻的大小从而影响着电池的电效率和热效率。 因此,对于活性层-集流体的粘接强度进行力学特征的测试和表征显得尤为重要,而这也将会给锂离子电池的力学建模提供重要的实验参数。\\
\indent 另一方面,在锂离子电池的充放电循环中,锂离子在正负极涂层材料中的脱嵌和嵌入会产生相当大的体积变化,这有可能会导致活性颗粒和粘结剂的脱层失效和电极材料的断裂,从而进一步引起电池容量的下降和电池的老化甚至失效。 从微观上预测和分析锂离子脱嵌和嵌入过程即不同SOC下的活性材料的力学特性对于了解锂离子电池容量消退的微观机理乃至建立多场耦合的预测模型都有着十分深远的意义。\\
\indent 本文将采用一种新的实验方法对集流体-活性层界面粘接强度在混合拉伸/剪切加载下进行直接测试,并研究其动态加载下的应变率效应。 另外,应用DFT计算研究了锂离子在磷酸亚铁锂晶体中的迁移过程并研究其弹性常数在不同SOC下的变化, 应用MD分子模拟的方法研究了锂离子扩散前后其晶体的拉伸-失效力学响应。
从而从宏观力学测试和微观分子模拟的角度对于电极界面强度乃至界面力学、电化学行为进行了全面和深入的研究。

% 工作概述

\end{cabstract}

% 如果习惯关键字跟在摘要文字后面,可以用直接命令来设置,如下:
% \ckeywords{\TeX, \LaTeX, CJK, 模板, 论文}
\begin{eabstract}
Fracture of the electrode material is one of the main degradation
mechanisms in Li-ion batteries, which causes the loss of electric
contact as well as enhances side reactions such as solid electrolyte interface (SEI) formation and dissolution due to the generation of new
interfaces. The electrode fracture may occur at various size scales, including crystals, polycrystals, and aggregates. And the deformation of
active particles can build a stress field in the crystalline
particles, thus
causing fracture at the level of aggregates such as the
debonding of binder and particles. To capture the failure of the constituent materials of the electrode, we must take into account
the mixture of several components with different sizes and properties and the multi-field coupling effect. It is important to understand the interfacial interactions inside the
constituent materials in order to mitigate the undesirable failure.\\
\indent On one hand, the adhesion between electrode and substrate also plays a number of significant roles in battery performance. The quality
of adhesion with the substrate will often dictate whether the electrochemically-induced
strains within the electrode materials are sufficient to cause failure/cracking. In addition,
the adhesion quality between electrode and substrate (which serves as the current collector)
will also influence the contact resistance and, thus electrical efficiency, of the system. Consequently, it is of vital importance to perform he testing and characterization of coating adhesion strength in cathode/anode, and the work will also provide sufficient information to  the mechanical modeling of lithium-ion batteries. \\
\indent  On the other hand, during battery cycling, lithium ions diffuse into and out of cathode/anode materials, causing large volume change. Large volume expansions and contractions in cathode/anode films can cause debonding failure,
significant cracking, capacity loss and degradation or failure. The modeling and predicting of the mechanical properties of active matter in the process of lithium intercalation and deintercalation, namely, different SOCs, is of vital importance, which can help reveal the micro mechanism of capacity fade and establish a multi-physics model.\\
\indent In the present paper, a new test method is proposed to
realize direct measurement of the adhesion strength of the
electrode under a combined tension/shear loading for different
stress states. And the dynamic test considering the loading rate effect is also conducted. Moreover, Li-ion diffusion in $LiFePO_4$ and the elastic constants in different SOCs are studied by first principles density function theroy(DFT) and the mechanical response untill fracture failure of $LiFePO_4$ before and after the Li-ion diffusion is studied via Molecular Dynamics Simulations. In summary, a multi-scale and comprehensive study of the mechanical and electrochemical properties of interfaces in lithium-ion batteries via mechanical testing ways and molecular simulation.
\end{eabstract}

