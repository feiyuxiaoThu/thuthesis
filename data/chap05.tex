\chapter{总结和展望}
\section{结果总结}
本研究对于锂离子电池活性层-集流体脱层失效行为进行了研究,对界面粘接强度进行了不同应变率状态下的混合拉伸-剪切实验,得到了锂离子电池力学建模中的界面力学参数。 研究发现:
\begin{itemize}
	\item 静态实验中,正极和负极的失效强度均在0.9$MPa$到2.4$MPa$之间,而相比之下,正极的强度要相对强于负极,这和正极的活性层的组成成分和涂布厚度有着直接关系。
	\item 动态加载下,正极和负极界面的断裂强度均有着显著的提升。以负极为例,相比静态实验中的断裂强度,其断裂强度在$0.1m/s$和$1m/s$速度加载下分别达到了$3.5MPa$和$3.8MPa$。
	\item 随着加载角度的变化即应力加载加载状态的变化,断裂强度发生了复杂的变化,但是统一而言,在有着较大剪切分量的情况下界面粘接强度会有所上升。 另外,尽管断裂强度的增加在较低的加载速度($0.1m/s$)处接近了阈值,但是断裂强度随着加载角度的变化规律却显著地受到加载速度的影响。
\end{itemize}
另一方面,本文也对电池活性材料的重要成分$LiFePO_4$晶体在锂离子的扩散过程中的力学乃至电化学性质利用第一性原理模拟和分子动力学计算的方法进行了模拟研究,研究发现:
\begin{itemize}
	\item 对于B和C两个晶体方向的扩散研究发现,锂离子在B方向输运的能垒显著低于C方向的能垒(约为12\%),锂离子的输运有着极强的方向性。 
	\item 对于锂离子在输运过程中的晶体构型弹性常数的计算发现,锂离子的输运过程中晶体的力学性质有着显著的改变,其杨氏模量可以最高有23\%的减小。 同时,由于晶体的构型和锂离子输运的影响,其力学参数有很强的各向异性,并会随着输运过程的进行而变化。
	\item 对$LiFePO_4$晶体在扩散前后(不同SOC下)的应力-应变响应的分子动力学模拟表明,晶体会在形变达到7\%左右时发生断裂失效,而锂离子的扩散前后会造成0.5\%左右的断裂应变差异。 这一模拟结果表明随着锂离子的脱嵌和嵌入所产生体积变化而产生的内部应力场的确会引起活性材料的断裂进而引起电极性能的消退。
\end{itemize}
\section{展望}
本文作者对于未来的工作研究的方向有一下几点展望:
\begin{enumerate}
	\item 本文分别从宏观实验和微观模拟的角度对于电极活性层内部的脱层断裂失效进行了测试和模拟研究,得到了详实的结果和数据。 本文的宏观实验数据可以给电池力学模型提供标定参数,但是所标定的模型往往是一个不基于电池实际组成的宏观力学模型,其在考虑多物理场的分析计算和对于电池内部诸多力学电化学过程的机理研究中显得力不从心。 在这一方面,未来可以尝试在满足一定的假设和边界条件下,直接用满足电池内部实际材料构成的微观或者介观的模型进行力学、化学乃至热学方面的建模,从而进行多物理场多尺度的耦合分析。
	\item 电池的胶接强度除了受到电池循环中锂离子的脱嵌和嵌入的影响,还必然会受到随着电池的老化而产生的一系列诸如气体产生等一系列物理化学过程的影响,对于电池工作循环中的实际特性表现和发展的研究中,如果能基于分子层次的模拟,建立微观模型,对于研究内部多物理过程的耦合乃至提升电池性能和指导电池设计都有着重要意义。
\end{enumerate}