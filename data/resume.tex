

\resumeitem{个人简历}

1997 年 2 月 3 日出生于陕西省汉中市。

2014 年 9 月考入清华大学航天航空学院工程力学系钱学森力学班,2018 年 7 月本科毕业并获得工学学士学位。





  

\researchitem{} 

I would like to dedicate this thesis to the respectful physicist and pioneer Dirac, whose book of Quantum Mechanics led me to the universe of science and more importantly the way how should I think over the miracle and explore the unknown of the world  . And I would like to quote a paragraph of his interview\footnote{T. Kuhn, interview with P.A.M. Dirac, 6 May 1963-Tape 62b, Niels Bohr Library, American Institute of Physics, New York.} which is an abstract of my principles:
\textit{I owe a lot to my engineering training because it taught me to tolerate approximations. Previously to that I thought...one should just concentrate on exact equations all the time. Then I got the idea that in the actual world all our equations are only approximate. We must just tend to greater and greater accuracy. In spite of the equations being approximate, they can be beautiful.}



